\documentclass[a4paper]{article}
\usepackage{amsmath}

\setlength{\parskip}{\smallskipamount}
\setlength{\parindent}{0pt}

\begin{document}

\title{Sistem Fluida dan Termal}
\author{}
\maketitle

\section{Pemodelan Sistem Cairan atau Likuid}

Pada sistem cairan (atau hidraulik), medium untuk transmisi energi adalah cairan.
Kita akan membahas mengenai elemene-elemen sistem dinamik dasar untuk cairan.

Pada sistem cairan, laju aliran volume (\textit{volume flow rate}) akan digunakan.
Besaran ini dilambangkan dengan $q_{v}$.

Pada sistem pneumatik, laju alirah massa (\textit{mass flow rate}) yang akan digunakan.
Besaran ini dilambangkan dengan $q_{m}$.

Hubungan antara $q_{v}$ dan $q_{m}$ dapat dinyatakan dalam persamaan berikut:
\begin{equation*}
q_m = \rho q_v
\end{equation*}
dengan $\rho$ adalah kerapatan massa. Hal ini dapat dilihat dari definisi
\begin{equation*}
q_v = \frac{\Delta vol}{\Delta t} = \frac{A \, \Delta x}{\Delta t} = v \, A
\end{equation*}
dan
\begin{equation*}
q_m = \frac{\Delta m}{\Delta t} = \rho \frac{\Delta vol}{\Delta t} = \rho \, q_v
\end{equation*}
di mana $vol$ menyatakan volume,
$\Delta x$: kecepatan elemen fluida,
$\Delta x$ adalah jarak yang ditempuh oleh elemen fluida,
$A$: luas daerah yang tegak lurus dengan arah aliran fluida.

\end{document}